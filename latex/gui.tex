\documentclass[]{beamer}
\usetheme{Dresden}
% \useoutertheme{split}

\usepackage{color}
\usepackage{graphicx}
\usepackage{listings}
\usepackage{lmodern} %% allow bold keywords
\usepackage{menukeys}
\usepackage{qtree}

\definecolor{darkgreen}{rgb}{0,0.5,0}
\definecolor{lightblue}{rgb}{0.2,0.2,1}

\lstset{language=Java,
	basicstyle=\ttfamily\footnotesize,
	keywordstyle=\color{purple},
	commentstyle=\color{darkgreen},
	numberstyle=\tiny\color{gray},
	stringstyle=\color{blue},
	tabsize=4,
	showstringspaces=false,
	breaklines=true,
	keepspaces=true,
	numbers=left,
	escapechar=@
}

\title{Java}
\subtitle{GUI}
\author{FSR Informatik}
\date{\today}

\begin{document}

\begin{frame}
\titlepage
\end{frame}

\begin{frame}{Overview}
\tableofcontents
\end{frame}

\section{GUI}
\subsection{Simple GUI}
\begin{frame}{}
	This lecture covers the basic principles of how to program Graphical User Interfaces (GUI) in Java.
	\vfill
	We will use a lightweight and simple package called \emph{Simple GUI}.
	You can download it here:\\
	\url{http://users.ifsr.de/~fredo/2013/Java/SimpleGui/simple_gui.jar}
\end{frame}

\begin{frame}{JAR}
	JAR stands for Java Archive. A \texttt{*.jar}-file \dots
	\begin{itemize}
		\item contains \texttt{*.class} files
		\item may contains \texttt{*.java} files
		\item may be digital signed
		\item may be compressed
	\end{itemize}
	\vfill
	\texttt{*.class} are the compiled Java classes.\\
	\texttt{*.java} are the sourcecode files.
	\vfill
	JARs often used as single file Java executables.
\end{frame}

\begin{frame}{Import JAR}
	\begin{enumerate}
		\item open \menu[,]{Project, Properties}
		\item select \keys{Java Build Path}
		\item select the tab \keys{Libaries}
		\item click \keys{Add External JARs\dots}
		\item choose the location to your JAR
	\end{enumerate}
	The imported package can be found on \emph{Referenced Libaries} in the Eclipse \emph{Package Explorer}.
	Your own \emph{default package} is located in \emph{src}.
	\vfill
	\emph{Do not move the JAR after the import. Eclipse will always reference the given location.}
\end{frame}

\begin{frame}{Javadoc}
	The Javadoc for Simple Gui is online:\\
	\url{http://users.ifsr.de/~fredo/2013/Java/SimpleGui/doc/}
	\vfill
	You see there are three classes and one interface.
	\begin{itemize}
		\item Interface ButtonConfiguration
		\item Class ButtonWindow
		\item Class DrawingWindow
		\item Class TextWindow
	\end{itemize}
\end{frame}

\subsection{Example TextWindow}
\begin{frame}[fragile]{The first Window}
	Windows in Java are treated as normal objects.
	\begin{lstlisting}
	import simple_gui.*;

	public class Example {

	    public static void main(String[] args) {
		
	        TextWindow window = new TextWindow();
	    }
	}	
	\end{lstlisting}
\end{frame}

\begin{frame}{The first Window - Screenshot}
	\includegraphics[scale=0.4]{res/gui_empty.png}
\end{frame}

\begin{frame}[fragile]{Hello World}
	\begin{lstlisting}
	import simple_gui.*;

	public class Example {

	    public static void main(String[] args) {
		
	        TextWindow window = new TextWindow();
	        
	        window.addOutputLine("Hello World!");
	        window.addOutputLine("");
	        window.addOutputLine("I am a Window.");
	    }
	}	
	\end{lstlisting}
\end{frame}

\begin{frame}{Hello World - Screenshot}
	\includegraphics[scale=0.4]{res/gui_textlines.png}
\end{frame}

\subsection{Input}
\begin{frame}[fragile]{Input - Processing - Output - Loop}
	\begin{lstlisting}
	public static void main(String[] args) {
		
	    TextWindow window = new TextWindow();
	    
	    while(true) {
	        if (window.isInputAvailable()) {
	            // input:
	            String str = window.getNextInputLine();
	            // processing:
	            str = "user: " + str;
	            // output:
	            window.addOutputLine(str);
	        }	    
	    }
	}	
	\end{lstlisting}
\end{frame}

\begin{frame}{}
	\includegraphics[scale=0.4]{res/gui_input.png}\\
	\footnotesize{During the second input. \\First input was \emph{Hello}. Second input will be \emph{World!}.}
\end{frame}

\begin{frame}[fragile]{Exit the Program}
	The \texttt{while(true)} statement implies the program will run forever. \\
	So you need a way to exit the program.
	\begin{lstlisting}
	while(true) {
	    if (window.isInputAvailable()) {
	        // input:
	        String str = window.getNextInputLine();
	        // processing:
	        if (str.compareTo("exit") == 0) {
	            window.close();
	            return;
	        }
	        str = "user: " + str;
	        // output:
	        window.addOutputLine(str);
	    }	    
	}
	\end{lstlisting}
\end{frame}

\subsection{Example ButtonWindow}
\begin{frame}[fragile]{Buttons}
	\begin{lstlisting}
	import simple_gui.*;

	public class ButtonTest {

	    public static void main(String[] args) {
		
	        // make a new window with 1 x 3 buttons
	        ButtonWindow window = new ButtonWindow(1, 3);
	    }
	}	
	\end{lstlisting}
\end{frame}

\begin{frame}{Buttons - Screenshot}
	We build a window with three buttons which have no function.
	\vfill
	\includegraphics[scale=0.4]{res/gui_buttons.png}
\end{frame}

%TODO maybe add graphic explaining this procedure
\begin{frame}{ButtonWindow}
	The Javadoc says \texttt{configureButton(int buttonNumber, ButtonConfiguration config)}
	adds some configuration to the button.
	\vfill
	\emph{ButtonConfiguration} is an Interface that describes two methods:
	\begin{itemize}
		\item \texttt{String getButtonText()}
		\item \texttt{void onClickAction()}
	\end{itemize}
	\vfill
	We have to write a class that implements \emph{ButtonConfiguration} and pass an 
	instance to the method \texttt{configureButton(\dots)}. \\
	The configured button will be named with the String the method \emph{getButtonText()} will return.
	A clicked configured button will call the method \emph{onClickAction()}.
\end{frame}

\begin{frame}[fragile]{CloseButton}
	\begin{lstlisting}[basicstyle=\ttfamily\scriptsize, escapechar=!]
	import simple_gui.*;

	public class CloseButton implements ButtonConfiguration {

	    @Override
	    public String getButtonText() {
	        return "Close";
	    }

	    @Override
	    public void onClickAction() {
	        // implementation follows
	    }
	}	
	\end{lstlisting}
\end{frame}

\begin{frame}[fragile]{CloseButton - Test}
	\begin{lstlisting}
	import simple_gui.*;

	public class ButtonTest {

	    public static void main(String[] args) {
		
	        // make a new window with 1 x 3 buttons
	        ButtonWindow window = new ButtonWindow(1, 3);
	        
	        // the numeration starts with 0
	        // so the right button is no. 2
	        window.configureButton(2, new CloseButton());
	    }
	}	
	\end{lstlisting}
\end{frame}

\begin{frame}{CloseButton - Screenshot}
	The button is successfully renamed.
	\vfill
	\includegraphics[scale=0.4]{res/gui_close.png}
\end{frame}

\begin{frame}[fragile]{CloseButton - onClickAction()}
	The method \texttt{window.close()} closes the window. 
	Unfortunately the class \emph{CloseButton} can not access the reference \texttt{window}.
	\begin{lstlisting}[basicstyle=\ttfamily\scriptsize, escapechar=!]
	@Override
	public String getButtonText() {
	    return "Close";
	}

	@Override
	public void onClickAction() {
	    // ...
	}
	\end{lstlisting}
\end{frame}

\begin{frame}[fragile]{CloseButton - onClickAction()}
	The class receives the reference through the constructor.
	\begin{lstlisting}[basicstyle=\ttfamily\scriptsize, escapechar=!]
	private ButtonWindow window;
	
	public CloseButton(ButtonWindow window) {
	    this.window = window;
	}
	
	@Override
	public String getButtonText() {
	    return "Close";
	}

	@Override
	public void onClickAction() {
	    this.window.close();
	}
	\end{lstlisting}
\end{frame}

\begin{frame}[fragile]{CloseButton - Final Test}
	\begin{lstlisting}[basicstyle=\ttfamily\scriptsize]
	import simple_gui.*;

	public class ButtonTest {

	    public static void main(String[] args) {
		
	        // make a new window with 1 x 3 buttons
	        ButtonWindow window = new ButtonWindow(1, 3);
	        
	        // pass the reference window as an argument
	        window.configureButton(2, new CloseButton(window));
	    }
	}	
	\end{lstlisting}
\end{frame}

\subsection{Example AbstractButton}
\begin{frame}[fragile]{AbstractButton}
	\begin{lstlisting}[basicstyle=\ttfamily\scriptsize, escapechar=!]
	import simple_gui.*;

	public abstract class AbstractButton implements ButtonConfiguration {
	
	    protected ButtonWindow window;
	    private String label;
	
	    public AbstractButton(ButtonWindow window, String label) {
	        this.window = window;
	        this.label = label;
	    }

	    @Override
	    public String getButtonText() {
	        return label;
	    }
	}	
	\end{lstlisting}
\end{frame}

\begin{frame}[fragile]{CloseButton}
	\begin{lstlisting}[basicstyle=\ttfamily\scriptsize, escapechar=!]
	import simple_gui.*;
	
	public class CloseButton extends AbstractButton {
	
	    public CloseButton(ButtonWindow window, String label) {
	        super(window, label);
	    }

	    @Override
	    public void onClickAction() {
	        this.window.close();
	    }
	}	
	\end{lstlisting}
\end{frame}

\begin{frame}[fragile]{Test}
	\begin{lstlisting}[basicstyle=\ttfamily\scriptsize]
	import simple_gui.*;

	public class ButtonTest {

	    public static void main(String[] args) {
		
	        // make a new window with 1 x 3 buttons
	        ButtonWindow window = new ButtonWindow(1, 3);
	        
	        window.configureButton(2, 
	            new CloseButton(window, "Close"));
	    }
	}	
	\end{lstlisting}
\end{frame}
\end{document}