%document
\documentclass[10pt]{beamer}
%theme
\usetheme{metropolis}
% packages
\usepackage{color}
\usepackage{listings}
\usepackage[ngerman]{babel}
\usepackage[utf8]{inputenc}
\usepackage{multicol}


% color definitions
\definecolor{mygreen}{rgb}{0,0.6,0}
\definecolor{mygray}{rgb}{0.5,0.5,0.5}
\definecolor{mymauve}{rgb}{0.58,0,0.82}

\lstset{language=Java,
	basicstyle=\ttfamily\footnotesize,
	keywordstyle=\color{purple},
	commentstyle=\color{darkgreen},
	numberstyle=\tiny\color{gray},
	stringstyle=\color{blue},
	tabsize=4,
	showstringspaces=false,
	breaklines=true,
	keepspaces=true,
	numbers=left,
	escapechar=@
}

\def\ContinueLineNumber{\lstset{firstnumber=last}}
\def\StartLineAt#1{\lstset{firstnumber=#1}}
\let\numberLineAt\StartLineAt



\newcommand{\codeline}[1]{
	\alert{\texttt{#1}}
}

% This Document contains the information about this course.

% Authors of the slides
\author{Felix Döring, Felix Wittwer}

% Name of the Course
\institute{Java-Kurs}

% Fancy Logo
\titlegraphic{\hfill\includegraphics[height=1.25cm]{../templates/fsr_logo_cropped}}


\title{Java}
\subtitle{Abstract}
\date{\today}

\begin{document}

\begin{frame}
	\titlepage
\end{frame}

\section{Abstract}
\subsection{}
\begin{frame}[fragile]{Abstract Class}
	The keyword \textbf{abstract} denotes an abstract class.
	\vfill
	\begin{lstlisting}
	public abstract class AbstractExample {
	
	}	
	\end{lstlisting}
	\vfill
    \begin{itemize}
	\item You can not create objects from an abstract class.\\
	\item Abstract classes can extend other abstract classes and can implement interfaces \footnote[1]{Interfaces will be discussed later}.\\
	\item Abstract classes can be extended by normal and abstract classes.
    \end{itemize}
\end{frame}

\begin{frame}[fragile]{Methods}
	An abstract class may has concrete methods and may has abstract methods.
	\begin{lstlisting}
	public abstract class AbstractExample {
	
	    public void printHello() {
	        System.out.println("Hello");	    
	    }
	    
	    public abstract String getName();
	}	
	\end{lstlisting}
	An abstract method forces the class to be abstract as well. \\
	%\emph{Methods in an interface are also abstract but not denoted as such.}
\end{frame}

\begin{frame}[fragile]{Subclasses}
	The subclass has to implement abstract methods or has to be abstract as well.
	All concrete methods will be regular inherited.
	\begin{lstlisting}[escapechar=!]
	public class Example extends AbstractExample {
	    
	    @Override
	    public String getName() {
	        return "Example";	    
	    }
	}	
	\end{lstlisting}
\end{frame}

%TODO more text
\begin{frame}{Why using Abstract?}
	Abstract classes are used to minimize similar code in related classes.
\end{frame}

%TODO Overview
%\begin{frame}{Abstract Class vs. Interface}
%
%\end{frame}

%TODO Common Errors
%\begin{frame}{Static vs. Abstract Methods}
%	Do not mix up statitic and abstract methods.
%\end{frame}
\end{document}