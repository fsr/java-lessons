\documentclass[]{beamer}
\usetheme{Dresden}
% \useoutertheme{split}

\usepackage{color}
\usepackage{graphicx}
\usepackage{listings}
\usepackage{lmodern} %% allow bold keywords
\usepackage{menukeys}
\usepackage{qtree}

\definecolor{darkgreen}{rgb}{0,0.5,0}
\definecolor{lightblue}{rgb}{0.2,0.2,1}

\lstset{language=Java,
	basicstyle=\ttfamily\footnotesize,
	keywordstyle=\color{purple},
	commentstyle=\color{darkgreen},
	numberstyle=\tiny\color{gray},
	stringstyle=\color{blue},
	tabsize=4,
	showstringspaces=false,
	breaklines=true,
	keepspaces=true,
	numbers=left,
	escapechar=@
}

\title{Java}
\subtitle{Object-Oriented Programming}
\author{FSR Informatik}
\date{\today}

\begin{document}

\begin{frame}
\titlepage
\end{frame}
\begin{frame}{Overview}
\tableofcontents
\end{frame}

\section{Introduction}
\subsection{What is OOP}
\begin{frame}{}
	Java is an object-oriented programming\footnote{OOP} language. \\
	\vspace{1em}
	The world of java consists of interacting objects. Those objects have a \textbf{state}, 
	a \textbf{behavior} and (maybe) a \textbf{relationship} to each other.
\end{frame}

\begin{frame}{An Example}
	You want to program an enrollment system, for a programming course. \\
	\vspace{1em}
	Your classes are:\\
	\begin{description}
		\item[student] who wants to attend the course
		\item[lesson] which is a part of the course
		\item[time slice] when your lessons are
		\item[room] where your lessons take place
		\item[\dots]
	\end{description}
	The more you think about it, the more complex this program becomes.
	Focus on the relevant things.
\end{frame}
	
\begin{frame}{Terminology} % redundant explanation - always a good thing
	Java knows \textbf{Classes} and \textbf{Objects}. A class is like a blueprint for an object. 
	With one blueprint you can build multiple objects of the same kind. \\
	\vspace{1em}
	For example: the class \emph{Student}\footnote{Remember: class names always start with a capital letter.} describes a student in general. 
	The objects - build from this class - describe each student in particular. \\
	Each (real-life) student has one object representing him in the enrollment system.
\end{frame}


\section{Classes and Objects}
\subsection{Class}
\begin{frame}{Class Structure}
	A class contains \textbf{attributes}. Those hold the state of the object built from this class. 
	A student may have the attributes \emph{name} and \emph{matriculation number}. \\
	\vspace{1em}
	A class also contain \textbf{methods}. Those define the abilities of the object built from this class.
	A student is able to \emph{print his timetable}. Maybe a student have methods to change some of his attributes.
\end{frame}

\begin{frame}[fragile]{Class \emph{Student}}
	\begin{lstlisting}
	public class Student {
	
	    // attributes:
	    String name;
	    int matriculationNumber;	
	
	    // method 1
	    public void printTimetable() {
	        System.out.println("my timetable");
	    }
	    
	    // method 2
	    public void changeName(String name) {
	    }
	}
	\end{lstlisting}
\end{frame}

\begin{frame}[fragile]{Class \emph{Student} - head}
	\begin{lstlisting}
	public class Student {
	
	}
	\end{lstlisting}
	This is the head of the class. \\
	The word \textbf{public} states the visibility of the class.
	We discuss this issue later. The word \textbf{class} marks this whole structure as a class.
	\emph{Student} is the name of the class. \\
	The curly brackets do not belong to the class head, but encapsulate the body.
\end{frame}

\begin{frame}[fragile]{Class \emph{Student} - body}
	\begin{lstlisting}
	    // attributes:
	    String name;
	    int matriculationNumber;	
	
	    // method 1
	    public void printTimetable() {
	        System.out.println("my timetable");
	    }
	    
	    // method 2
	    public void changeName(String name) {
	    }
	\end{lstlisting}
	This is the body of the class. The body contains the \textbf{attributes} and the \textbf{methods}.
\end{frame}

\begin{frame}[fragile]{Class \emph{Student} - attributes}
	\begin{lstlisting}
	    // attributes:
	    String name;
	    int matriculationNumber;	
	\end{lstlisting}
\end{frame}

\begin{frame}[fragile]{Class \emph{Student} - methods}
	\begin{lstlisting}
	    // method 1
	    public void printTimetable() {
	        System.out.println("my timetable");
	    }
	    
	    // method 2
	    public void changeName(String name) {
	    }
	\end{lstlisting}
	Inside a method are the instructions for the program. Always remember that. 
	\emph{A common mistake is to place instructions outside of a method.}
\end{frame}

\subsection{Object}

\begin{frame}[fragile]{Creation}
	We learned how to declare and assign a primitive datatype.
	\begin{lstlisting}
	    int a; // declare a
	    a = 273; // assign 273 to a
	\end{lstlisting} 
	The creation of an object works similar.
	\begin{lstlisting}
	    Student example; // declare example
	    example = new Student(); // create an instance of Student
	\end{lstlisting}
	The \textbf{object} derived from a \textbf{class} is also called \textbf{instance}.
	The variable\footnote{in this listing \emph{example}} is called the \textbf{reference}.
\end{frame}

\section{Methods}
\subsection{Methods}
\begin{frame}[fragile, allowframebreaks]{Calling a Method}
	\begin{lstlisting}
	public class Student {
	
	    public void printTimetable() {
	        System.out.println("my timetable");
	    }
	    
	    public void printName() {
	        System.out.println("Jane");
	    }
	}
	\end{lstlisting}
	The class \emph{Student} has two methods: \emph{void printTimetable()} and \emph{void printName()}.
	\begin{lstlisting}
	public class EnrollmentSystem {
	    
	    public static void main(String[] args) {
	        System.out.println("hello");
	        Student example = new Student(); // creation
	        example.printTimetable(); // method call
	    }
	}
	\end{lstlisting}
	You can call a method of an object after its creation with \textbf{reference.methodName()}.
\end{frame}

\subsection{Main Method}
\begin{frame}[fragile]{Main Method}
	Which line will be printed?
	\begin{lstlisting}
	public class EnrollmentSystem {
	
	    public void test() {
	        System.out.println("test");
	    }
	    
	    public static void main(String[] args) {
	        System.out.println("hello");
	    }
	}
	\end{lstlisting}
\end{frame}

\begin{frame}[fragile]{Main Method}
	Only the line \emph{hello} will be printed. 
	Each Java program starts in the \textbf{main method}. 
	Other methods have to be called.
	\begin{lstlisting}
	public class EnrollmentSystem {
	
	    public void test() {
	        System.out.println("test");
	    }
	    
	    public static void main(String[] args) {
	        System.out.println("hello");
	    }
	}
	\end{lstlisting}
\end{frame}

\subsection{Arguments}
\begin{frame}[fragile]{Methods with Arguments}
	You can call a method with e.g. two arguments via \texttt{methodName(arg1, arg2)}.
	\begin{lstlisting}
	public class Calc {
	
	    public void add(int summand1, int summand2) {
	        System.out.println(summand1 + summand2);
	    }
	    
	    public static void main(String[] args) {
	        int summandA = 1;
	        int summandB = 2;
	        Calc calculator = new Calc();
	        System.out.print("1 + 2 = ");
	        calculator.add(summandA, summandB); // prints: 3
	    }
	}
	\end{lstlisting}
\end{frame}

\begin{frame}[fragile]{Methods with Arguments - another Example}
	\begin{lstlisting}
	public class Calc {
	
	    public void concatenate(String base, int supplement) {
	        System.out.println(base + supplement);
	    }
	    
	    public static void main(String[] args) {
	        String part1 = "1";
	        int part2 = 2;
	        Calc calculator = new Calc();
	        System.out.print("1 + 2 = ");
	        calculator.concatenate(part1, part2); // prints: 12
	    }
	}
	\end{lstlisting}
\end{frame}

\begin{frame}[fragile]{Methods with Arguments - yet another Example}
	\begin{lstlisting}
	public class Calc {
	
	    public void concatenate(String base, int supplement) {
	        System.out.println(base + supplement);
	    }
	    
	    public static void main(String[] args) {
	        Calc calculator = new Calc();
	        System.out.print("1 + 2 = ");
	        calculator.concatenate("1", 2); // prints: 12
	    }
	}
	\end{lstlisting}
\end{frame}

\subsection{Return Value}
\begin{frame}[fragile]{Methods with Return Value}
	A method without a return value is indicated by \textbf{void}:
	\begin{lstlisting}
	public void add(int summand1, int summand2) {
	    System.out.println(summand1 + summand2);
	}
	\end{lstlisting}
	A method with an \textbf{int} as return value:
	\begin{lstlisting}
	public int add(int summand1, int summand2) {
	    return summand1 + summand2;
	}
	\end{lstlisting}
	%% TODO explain return statement
\end{frame}

\begin{frame}[fragile]{Calling Methods with a return value}
	\begin{lstlisting}
	public class Calc {
	
	    public int add(int summand1, int summand2) {
	        return summand1 + summand2;
	    }
	    
	    public static void main(String[] args) {
	        Calc calculator = new Calc();
	        int sum = calculator.add(3, 8);
	        System.out.print("3 + 8 = " + sum); // prints: 3 + 8 = 11
	    }
	}
	\end{lstlisting}
\end{frame}

\begin{frame}[fragile]{Nested Calls}
	You can nest your method calls.
	\begin{lstlisting}
	public class Calc {
	
	    public int add(int summand1, int summand2) {
	        return summand1 + summand2;
	    }
	    
	    public static void main(String[] args) {
	        Calc calc = new Calc();
	        int sum = calc.add(3, calc.add(8, 5));
	        System.out.print("3 + 8 + 5 = " + sum); 
	        // prints: 3 + 8 + 5 = 16
	    }
	}
	\end{lstlisting}
\end{frame}

\section{Working with Objects}
\subsection{Attributes}
\begin{frame}[fragile]{Attributes in a Class}
	An \textbf{attribute} is similar to a variable, but declared outside a method\footnote{But still inside the class body} 
	and hence accessible by all methods of the class.
	The attributes represent the state of an object.
	\begin{lstlisting}
	public class Student {
	
	    public String name;
	    
	    public void printName() {
	        System.out.println(name);
	    }
	}
	\end{lstlisting}
\end{frame}

\begin{frame}[fragile]{Direct Access}
	You can access an attribute directly using the \textbf{dot operator}\footnote{\textbf{+ - . =} are called operators} without round brackets.
	\begin{lstlisting}
	public class EnrollmentSystem {
	    
	    public static void main(String[] args) {
	        Student example = new Student();
	        example.name = "Jane Doe";
	        example.printName();
	    }
	}
	\end{lstlisting}
\end{frame}

\begin{frame}[fragile]{Multiple Objects}
	Multiple objects may differ in the value of their attributes.
	\begin{lstlisting}
	public class EnrollmentSystem {
	    
	    public static void main(String[] args) {
	        Student exampleA = new Student();
	        Student exampleB = new Student();
	        exampleA.name = "Jane Doe";
	        exampleB.name = "Julius Doe";
	        exampleA.printName();
	        exampleB.printName();
	    }
	}
	\end{lstlisting}
\end{frame}

\subsection{this}
\begin{frame}[fragile]{this}
	The keyword \textbf{this} is a \textbf{reference} to the current object. 
	So you can access the object from inside the object.
	\begin{lstlisting}
	public class Student {
	
	    public String @\textcolor{gray}{\texttt{name}}@;
	    
	    public void changeName(String @\textcolor{red}{\texttt{name}}@) {
	        this.@\textcolor{gray}{\texttt{name}}@ = @\textcolor{red}{\texttt{name}}@;   
	    }
	}
	\end{lstlisting}
	The local scope of \textcolor{red}{\texttt{name}} overrides the class scope of \textcolor{gray}{\texttt{name}}. 
	Without \textbf{this} you can not access the variable \textcolor{gray}{\texttt{name}} in this specific method.
\end{frame}

\begin{frame}[fragile, allowframebreaks]{Calling a Method inside a Class}
	\begin{lstlisting}
	public class Student {
	
	    public void printTimetable() {
	        this.printName(); // call another method
	        System.out.println("my timetable");
	    }
	    
	    public void printName() {
	        System.out.println("Jane");
	    }
	}
	\end{lstlisting}
	A method can call other methods inside the same class.
	\begin{lstlisting}
	public class EnrollmentSystem {
	    
	    public static void main(String[] args) {
	        System.out.println("hello");
	        Student example = new Student(); // creation
	        example.printTimetable(); // method call
	    }
	}
	\end{lstlisting}
\end{frame}

\subsection{Constructor}
\begin{frame}[fragile]{Constructor}
	The \textbf{constructor} is a special method, which is invoked while creating the object. 
	This method has the same name as the class. It is often used to initialize the attributes.
	The constructor has no return value. 
	This \textbf{must not} be indicated with void.
	\begin{lstlisting}	
	public class Student {
	
	    public String name;
	    
	    public Student() {
	        this.name = "Jane Doe"; 
	    }
	}
	\end{lstlisting}	
\end{frame}
\begin{frame}[fragile]{Call the Constructor}
	\begin{lstlisting}	
	public class EnrollmentSystem {
	    
	    public static void main(String[] args) {
	        Student example = new Student();
	        System.out.println(example.name);
	        // prints: Jane Doe
	    }
	}
	\end{lstlisting}	
\end{frame}
\begin{frame}[fragile]{Constructor with arguments}
	A constructor can take arguments.
	\begin{lstlisting}	
	public class Student {
	
	    public String name;
	    
	    public Student(String name) {
	        this.name = name;	    
	    }
	}
	\end{lstlisting}	
\end{frame}
\begin{frame}[fragile]{Call the Constructor with arguments}
	\begin{lstlisting}	
	public class EnrollmentSystem {
	    
	    public static void main(String[] args) {
	        Student exampleA = new Student("Jane Doe");
	        Student exampleB = new Student("Julius Doe");
	    }
	}
	\end{lstlisting}	
\end{frame}


% \section{Conclusion} %%% TODO

%% repetition: an obj can be used as complex data structure vs. primitive
	%% like student data as a student
%% most of the time objects are actors. 
%% The also represent non touchable things in world. Such as: Time, Weather
%% OBJ used to model abstract things:
	%% GUI: window, button, etc.
	%% Role, Supervisor

%% TODO avoid talking about references without explaining them	
%% TODO slide about references (maybe in a later slide set):
%% Every object is handled by Java through a reference
%% This means, when you move or assign an object, only the reference gets moved, 
%% the object stays in the same position in your computers memory.
%% if a reference is shared, everyone who shares it, influences the same object! No copies are made!
%%
%% Example:
%% Student student1 = new Student();
%% Student student2 = student1; // share reference of student1
%% student1.name = "Peter"; // influence the object
%% System.out.println(student2.name); // prints "Peter"
%%
%% Also: draw a sketch.



\end{document}