\documentclass[]{beamer}
\usetheme{Dresden}
% \useoutertheme{split}

\usepackage{color}
\usepackage{graphicx}
\usepackage{listings}
\usepackage{lmodern} %% allow bold keywords
\usepackage{menukeys}
\usepackage{qtree}

\definecolor{darkgreen}{rgb}{0,0.5,0}
\definecolor{lightblue}{rgb}{0.2,0.2,1}

\lstset{language=Java,
	basicstyle=\ttfamily\footnotesize,
	keywordstyle=\color{purple},
	commentstyle=\color{darkgreen},
	numberstyle=\tiny\color{gray},
	stringstyle=\color{blue},
	tabsize=4,
	showstringspaces=false,
	breaklines=true,
	keepspaces=true,
	numbers=left,
	escapechar=@
}

\title{Java 03}
\subtitle{Inheritance}
\author{FSR Informatik}
\date{\today}

\begin{document}

\begin{frame}
\titlepage
\end{frame}
\begin{frame}{Overview}
\tableofcontents
\end{frame}

\section{Control Structures}
\subsection{boolean}
\begin{frame}[fragile]{Statements}
	A statement can be \textbf{true} or \textbf{false}. 
	You can use a boolean variable to save the value of a statement.
	\begin{lstlisting}
	boolean x1 = (17 < 20);  // true
	boolean x2 = (17 >= 20); // false
	boolean x3 = (17 == 20); // false
	boolean x4 = (17 != 20); // true
	\end{lstlisting}
\end{frame}

\begin{frame}{Relational Operators}
	Some relational operators for your statements:
	\begin{description}
		\item[A $<$ B] A smaller B
		\item[A $<=$ B] A smaller or equal B
		\item[A $>$ B] A greater B
		\item[A $>=$ B] A greater or euqal B
		\item[A $==$ B] A equal B
		\item[A $!=$ B] A not equal B
	\end{description}
\end{frame}

\begin{frame}{Logic}
	You can combine statements with logic operators like \textbf{NOT}, \textbf{AND} and \textbf{OR}.
	\begin{description}
		\item[!A] not A - is true if A is false
		\item[A \&\& B] A and B - is true if both statements are true
		\item[A \textbar\textbar\ B] A or B - is true if one statement is true or both
	\end{description}
\end{frame}

\begin{frame}[fragile]{Examples}
	\begin{lstlisting}
	boolean x1 = (17 < 20) && (4 < 16);  // true
	boolean x2 = (17 >= 20) || (4 < 16); // true
	boolean x3 = (17 == 20) && (4 == 4); // false
	boolean x4 = !(17 != 20);            // false
	boolean x5 = !(4 == 16) || (((17 == 20) && (4 == 4))); 
	// x5 is true
	\end{lstlisting}
\end{frame}

\subsection{condition}
\begin{frame}[fragile]{if}
	%% grams is AE and grammes is BEsymbol
	\begin{lstlisting}
	public class PostOffice {
	
	    public static void main(String[] args) {
	
	        int letterWeight = 46; // in grams
	        int postage = 90; // in ct
	    
	        if(letterWeight <= 20) {
	            postage = 60;
	        }
	    }
	}
	\end{lstlisting}
\end{frame}

\begin{frame}[fragile]{if}
	%TODO red source code with same format as normal source code
	% if some is able to do that, pleas mail me
	If the \textcolor{red}{statement} is true the \textcolor{blue}{body} inside the curly brackets will be executed.
	If the \textcolor{red}{statement} is false the \textcolor{blue}{body} will not be executed.
	\begin{lstlisting}
	int letterWeight = 53; // in grams
	int postage = 90; // in ct
	    
	if(@\textcolor{red}{letterWeight <= 20}@) {
	    @\textcolor{blue}{postage = 60;}@
	}
	\end{lstlisting}
\end{frame}

\begin{frame}[fragile]{if - example}
	\begin{lstlisting}
	public class PostOffice {
	
	    public static void main(String[] args) {
	
	        int letterWeight = 46;
	        int postage = 90;
	        
	        if(letterWeight <= 20) { // false
	            postage = 60;
	        }
	        
	        System.out.println(postage + "ct");
	        // prints: 90ct
	    }
	}
	\end{lstlisting}
	%\emph{If not neceassary the class head will not shown in the following examples.}
\end{frame}

\begin{frame}[fragile]{if - counter example}
	\begin{lstlisting}
	public class PostOffice {
	
	    public static void main(String[] args) {
	
	        int letterWeight = 17;
	        int postage = 90;
	    
	        if(letterWeight <= 20) { // true
	            postage = 60;
	        }
	        
	        System.out.println(postage + "ct");
	        // prints: 60ct
	    }
	}
	\end{lstlisting}
\end{frame}

\subsection{else}
\begin{frame}[fragile]{else}
	\begin{lstlisting}
	public class PostOffice {
	
	    public static void main(String[] args) {
	
	        int letterWeight = 17;
	        int postage = 0;
	    
	        if(letterWeight <= 20) { // true
	            postage = 60;
	        } else {
	            postage = 90;
	        }
	        
	        System.out.println(postage + "ct");
	        // prints: 60ct
	    }
	}
	\end{lstlisting}
\end{frame}

\begin{frame}[fragile]{else if}
	\begin{lstlisting}
	public class PostOffice {
	
	    public static void main(String[] args) {
	
	        int letterWeight = 37;
	        int postage = 0;
	    
	        if(letterWeight <= 20) { // false
	            postage = 60;
	        } else if (letterWeight <= 50) { // true
	            postage = 90;
	        }
	        
	        System.out.println(postage + "ct");
	        // prints: 90ct
	    }
	}
	\end{lstlisting}
\end{frame}

\begin{frame}[fragile]{multiple else if}
	\begin{lstlisting}
	public static void main(String[] args) {
	
	    int letterWeight = 37;
	    int postage = 0;
	    
	    if(letterWeight <= 20) { // false
	        postage = 60;
	    } else if (letterWeight <= 50) { // true
	        postage = 90;
	    } else if (letterWeight <= 500 ) { // true
	        postage = 145;
	    }
	        
	    System.out.println(postage + "ct");
	    // prints: 90ct
	}
	\end{lstlisting}
\end{frame}

\begin{frame}{multiple else if}
	You can use as many \emph{else if} as you want.
	If multiple conditions are true, only the first one is relevant. \\
	\vfill
	\textbf{Warning: } Other programing languages may handle this case differently.
\end{frame}

\subsection{Loops}
\begin{frame}[fragile]{For Loop}
	The for loop starts with an assignment: \textcolor{gray}{\texttt{int i = 4}}. \\
	Every lap the \textcolor{blue}{body} will be executed and prints the changing variable i. \\
	After each lap the i will be incremented via \textcolor{orange}{\texttt{i++}}. \\
	The loop will stop if the condition \textcolor{red}{\texttt{i $<=$ 10}} becomes false. 
	It will never start if the condition is false at begin.
	\begin{lstlisting}
	public static void main(String[] args) {
	
	    for ( @\textcolor{gray}{int i = 4}@; @\textcolor{red}{i <= 10}@; @\textcolor{orange}{i++}@) {
	        @\textcolor{blue}{System.out.print(i + " ");}@   
	    }
	    //prints: 4 5 6 7 8 9 10
	}
	\end{lstlisting}
\end{frame}

\begin{frame}[fragile]{Endless Loop}
	If you need an endless loop. Use \textbf{for} with empty parameters.
	\begin{lstlisting}
	public static void main(String[] args) {
	
	    for (;;) {
	        System.out.println("I am still running");
	    }
	}
	\end{lstlisting}
\end{frame}

\begin{frame}[fragile]{While Loop}
	The while loop will be executed \emph{while} the \textcolor{red}{condition} is true.
	\begin{lstlisting}
	public static void main(String[] args) {
	
	    int i = 1;
	    while (@\textcolor{red}{i < 5}@) {
	        i++;
	        System.out.print(i + " ");
	    }
	    // prints: 2 3 4 5
	}
	\end{lstlisting}
\end{frame}

\begin{frame}[fragile]{Do-While Loop}
	The do-while loop will be executed until the \textcolor{red}{condition} becomes false.
	\begin{lstlisting}
	public static void main(String[] args) {
	
	    int i = 1;
	    do {
	        i++;
	        System.out.print(i + " ");
	    } while (@\textcolor{red}{i < 5}@);
	    // prints: 2 3 4 5
	}
	\end{lstlisting}
	\emph{Do not forget the semicolon at the end.}
\end{frame}

\begin{frame}{While vs. Do-While}
	There is a difference between the while and the do-while loop. \\
	\vfill
	If the loop condition false at start:
	\begin{itemize}
		\item the while loop will not start at all
		\item the do-while loop will run one time, if the condition stays false
	\end{itemize}
\end{frame}

\section{Array}
\subsection{Unidimensional Array}
\begin{frame}[fragile]{Array}
	An array is a data-type that can hold a \textbf{fixed number} of elements. 
	An Element can be any simple data-type or object.
	\begin{lstlisting}
	public static void main(String[] args) {
	
	    int[] intArray = new int[10];
	    intArray[8] = 7; // assign 7 to the 9th element
	    intArray[9] = 8; // assign 8 to the last element
	    
	    System.out.println(intArray[8]); // prints: 7
	}
	\end{lstlisting}
	You can access every element via an index. A n-element array has indexes from 0 to (n-1).
\end{frame}

\begin{frame}[fragile]{Array Initialization} % AE
	You can initialize an array with a set of elements.
	\begin{lstlisting}
	public static void main(String[] args) {
	
	    int[] intArray = {3, 2, 7};
	    
	    System.out.println(intArray[0]); // prints: 3
	    System.out.println(intArray[1]); // prints: 2
	    System.out.println(intArray[2]); // prints: 7
	}
	\end{lstlisting}
\end{frame}

\begin{frame}[fragile]{Alternative Declaration}
	There two possible positions for the square brackets. 
	%I would recommend the first version to improve readability.
	\begin{lstlisting}
	public static void main(String[] args) {

	    // version 1	
	    int[] intArray1 = new int[10];
	    
	    // version 2
	    int intArray2[] = new int[10];
	}
	\end{lstlisting}
\end{frame}

\subsection{Multi-Dimensional Array}
\begin{frame}[fragile]{2-Dimensional Array}
	Arrays work with more than one dimension. 
	An m-dimensional array has m indexes for one element.
	\begin{lstlisting}
	public static void main(String[] args) {

	    // an array with 100 elements
	    int[][] intArray = new int[10][10];
	    
	    intArray[0][0] = 0;
	    intArray[0][9] = 9;
	    intArray[9][9] = 99;
	}
	\end{lstlisting}
\end{frame}

\begin{frame}[fragile]{Assignment with Loops}
	Loops are often used to assign elements in arrays.
	\begin{lstlisting}
	public static void main(String[] args) {

	    int[][] intArray = new int[10][10];
	    
	    for(int i = 0; i < 10; i++) {
	        for(int j = 0; j < 10; j++) {
	            intArray[i][j] = i*10 + j;
	        }
	    }
	}
	\end{lstlisting}
\end{frame}

\section{Inheritance}
\subsection{Inheritance}
\begin{frame}{Inheritance}
	Objects in real world have a relationship to each other.
	In software we have to modell them if necessary. \\
	\vfill
	Java has a concept for one of those relationships: \textbf{Inheritance}.
	\vfill
	Inheritance is a modell of the \emph{is a kind of} - relation.
	Using this we can structure our classes.
\end{frame}

\begin{frame}[fragile]{A special Delivery}
	Our class \emph{Letter} is a kind of \emph{Delivery} denoted by the keyword \textbf{extends}.
	\begin{itemize}
		\item \emph{Letter} is a \textbf{subclass} of the class \emph{Delivery}
		\item \emph{Delivery} is the \textbf{superclass} of the class \emph{Letter}
	\end{itemize}
	\begin{lstlisting}
	public class Letter extends Delivery {
	
	}
	\end{lstlisting}
	\vfill
	As mentioned implicitly above a class can has multiple subclasses. 
	But a class can only inherit directly from one superclass.
\end{frame}

\begin{frame}[fragile]{Example}
	We have the classes: \emph{PostOffice}, \emph{Delivery} and \emph{Letter}.
	They will be used for every example in this section and they will grow over time.
	\begin{lstlisting}
	public class Delivery {
	
	    public String address;
	    public String sender;
	    
	    public void printAddress() {
	        System.out.println(this.address);
	    }
	}
	\end{lstlisting}
	
	\begin{lstlisting}
	public class Letter extends Delivery {
	
	}
	\end{lstlisting}
\end{frame}

\begin{frame}[fragile]{Inherited Attributes}
	The class \emph{Letter} inherits all attributes from the superclass \emph{Delivery}.
	\begin{lstlisting}
	public class PostOffice {
	
	    public static void main(String[] args) {
	    
	        Letter letter = new Letter();
	        
	        letter.address = "cafe ascii, Dresden";
	        
	        System.out.println(letter.address);
	        // prints: cafe ascii, Dresden
	    }	
	}
	\end{lstlisting}
\end{frame}

\begin{frame}[fragile]{Inherited Methods}
	The class \emph{Letter} also inherits all methods from the superclass \emph{Delivery}.
	\begin{lstlisting}
	public class PostOffice {
	
	    public static void main(String[] args) {
	    
	        Letter letter = new Letter();
	        
	        letter.address = "cafe ascii, Dresden";
	        
	        letter.printAddress();
	        // prints: cafe ascii, Dresden
	    }	
	}
	\end{lstlisting}
\end{frame}

\begin{frame}[fragile]{Override Methods}
	The method printAddress() is now additional definded in \emph{Letter}.
	\begin{lstlisting}[escapechar=!]
	public class Letter extends Delivery {
	
	    @Override
	    public void printAddress() {
	        System.out.println("a letter for " + this.address);    
	    }	
	}
	\end{lstlisting}
	% programer is AE
	\texttt{@Override} is an annotation. 
	It helps the programer to identify overwritten methods.
	It is not neccessary for running the code but improves readability.
	What annotations else can do we discuss in a future lesson.
\end{frame}
\begin{frame}[fragile]{Override Methods}
	Now the method \texttt{printAddress()} defined in \emph{Letter} will be used instead of the method defined
	in the superclass \emph{Delivery}.
	\begin{lstlisting}
	public class PostOffice {
	
	    public static void main(String[] args) {
	    
	        Letter letter = new Letter();
	        
	        letter.address = "cafe ascii, Dresden";
	        
	        letter.printAddress();
	        // prints: a letter for cafe ascii, Dresden
	    }	
	}
	\end{lstlisting}
\end{frame}

\subsection{Constructor}
\begin{frame}[fragile]{Super()}
	If we define a \textbf{constructor with arguments} in \emph{Delivery} we have to define a constructor
	with the same list of arguments in every subclass.
	\begin{lstlisting}[basicstyle=\ttfamily\scriptsize]
	public class Delivery {
	
	    public String address;
	    public String sender;
	    
	    public Delivery(String address, String sender) {
	        this.address = address;
	        this.sender = sender;
	    }
	    	    
	    public void printAddress() {
	        System.out.println(this.address);
	    }
	}
	\end{lstlisting}
\end{frame}

\begin{frame}[fragile]{Super()}
	For the constructor in the subclass \emph{Letter} we can use \texttt{super()} to call the constructor
	from the superclass.
	\begin{lstlisting}[escapechar=!]
	public class Letter extends Delivery {

	    public Letter(String address, String sender) {
	        super(address, sender);
	    }
	
	    @Override
	    public void printAddress() {
	        System.out.println("a letter for " + this.address);    
	    }	
	}
	\end{lstlisting}
\end{frame}

\begin{frame}[fragile]{Super() - Test}
	\begin{lstlisting}
	public class PostOffice {
	    
	    public static void main(String[] args) {	    
	        Letter letter = 
	            new Letter("cafe ascii, Dresden", "");
	        
	        letter.printAddress();
	        // prints: a letter for cafe ascii, Dresden
	    }
	}
	\end{lstlisting}
\end{frame}

\subsection{Implicit Inheritance}
\begin{frame}{Object}
	Every class is a subclass from the class \emph{Object}. 
	Therefore every class inherits methods from \emph{Object}.
	\vfill
	See \scriptsize\url{http://docs.oracle.com/javase/7/docs/api/java/lang/Object.html} \normalsize for
	a full reference of the class \emph{Object}.
\end{frame}

\begin{frame}[fragile]{toString()}
	\emph{Letter} is a subclass of \emph{Object}.
	Therefore \emph{Letter} inherits the method \texttt{toString()} from \emph{Object}.\\
	\texttt{System.out.println(argument)} will call \texttt{argument.toString()} to receive
	a printable String.
	\begin{lstlisting}[escapechar=!]
	public class PostOffice {
	    
	    public static void main(String[] args) {	    
	        Letter letter = 
	            new Letter("cafe ascii, Dresden", "");
	        
	        System.out.println(letter);
	        // prints: Letter@_some_HEX-value_
	        // for example: Letter@4536ad4d
	    }
	}
	\end{lstlisting}
\end{frame}

\begin{frame}[fragile]{Override toString()}
	\begin{lstlisting}[escapechar=!]
	public class Letter extends Delivery {

	    public Letter(String address, String sender) {
	        super(address, sender);
	    }
	
	    @Override
	    public String toString() {
	        return "a letter for " + this.address;
	    }	
	}
	\end{lstlisting}
\end{frame}

\begin{frame}[fragile]{Override toString() - Test}
	\begin{lstlisting}
	public class PostOffice {
	    
	    public static void main(String[] args) {	    
	        Letter letter = 
	            new Letter("cafe ascii, Dresden", "");
	        
	        System.out.println(letter);
	        // a letter for cafe ascii, Dresden
	    }
	}
	\end{lstlisting}
\end{frame}

%\section{Visability}
%\subsection{}
%\begin{frame}{}
%\end{frame}

\end{document}