%document
\documentclass[10pt]{beamer}
%theme
\usetheme{metropolis}
% packages
\usepackage{color}
\usepackage{listings}
\usepackage[ngerman]{babel}
\usepackage[utf8]{inputenc}
\usepackage{multicol}


% color definitions
\definecolor{mygreen}{rgb}{0,0.6,0}
\definecolor{mygray}{rgb}{0.5,0.5,0.5}
\definecolor{mymauve}{rgb}{0.58,0,0.82}

\lstset{language=Java,
	basicstyle=\ttfamily\footnotesize,
	keywordstyle=\color{purple},
	commentstyle=\color{darkgreen},
	numberstyle=\tiny\color{gray},
	stringstyle=\color{blue},
	tabsize=4,
	showstringspaces=false,
	breaklines=true,
	keepspaces=true,
	numbers=left,
	escapechar=@
}

\def\ContinueLineNumber{\lstset{firstnumber=last}}
\def\StartLineAt#1{\lstset{firstnumber=#1}}
\let\numberLineAt\StartLineAt



\newcommand{\codeline}[1]{
	\alert{\texttt{#1}}
}

% This Document contains the information about this course.

% Authors of the slides
\author{Felix Döring, Felix Wittwer}

% Name of the Course
\institute{Java-Kurs}

% Fancy Logo
\titlegraphic{\hfill\includegraphics[height=1.25cm]{../templates/fsr_logo_cropped}}


\title{Java}
\subtitle{Scopes \& GUI}
\date{\today}

\begin{document}

\begin{frame}
\titlepage
\end{frame}

\begin{frame}{Overview}
\tableofcontents
\end{frame}

\section{Scopes}
\subsection{Classes}

\begin{frame}[fragile]{Visiabilities}
	\begin{lstlisting}
	class MyGreatClass {
	
		//Attributes are public by default
		Car myCar; 
	
		//Public are available in every part of our code.
		public Cat myCat;
		
		//Private Attributes can only be acced via a method
		private House myHouse;
	}
		
	\end{lstlisting}
\end{frame}

\subsection{Controll Structures}
\begin{frame}[fragile]{If}
	\begin{lstlisting}
	...
	int a = 1;
	if(...) {
		int b = 5;
		System.out.println(a);
		System.out.println(b);
	}
	
	System.out.println(a);
	System.out.println(b);
	...
	\end{lstlisting}
	
	b is only available in the scope of the if.
	
	b is not outside of the if available.
	
	\color{red} WILL NOT COMPILE
\end{frame}

\begin{frame}[fragile]{For}
	\begin{lstlisting}
	for(int i = 0; i <= 100; i++) {
		int b = 3;
		System.out.println(i);
		System.out.println(b);
	}
	\end{lstlisting}
	
	b will be redefined in every round of the loop and is only available in the for loop.
	
	The scope is created at the begining and destroyed at the end of each round.
\end{frame}

\begin{frame}[fragile]{While}
	\begin{lstlisting}
	int i = 0;
	while(i <= 100) {
		int b = 3;
		System.out.println(i);
		System.out.println(b);
	}
	\end{lstlisting}
	
	For and while got the same scope behavior.
\end{frame}
  
\subsection{Mulit definitions}
\begin{frame}[fragile]{Examples}
	\begin{lstlisting}
	public class myClass {
		private int a;
	
		public myClass(int a) {
			this.a = a;
		}
	}
	\end{lstlisting}

	Use nearest definition.

	In one scope every variable name can be defined only one time.
\end{frame}

\begin{frame}{What we learned}
	Scopes are definition areas for variables.
	Every Block defines a new Scope.
\end{frame}
    
\section{GUI}
\subsection{Window}
\begin{frame}[fragile]{Window}
  Windows are created by creating a \texttt{JFrame} object.
	\begin{lstlisting}
		// Create a new window
		JFrame window = new JFrame();

		// Set its title and size
		window.setTitle("Guestbook");
		window.setSize(500, 500);
    
    // Show the window
    window.setVisible(true);
	\end{lstlisting}
\end{frame}

\begin{frame}[fragile]{Window}
  You can add panels and other elements to the window.
	\begin{lstlisting}
		// Add a blue background panel
		JPanel backgroundPanel = new JPanel();
		backgroundPanel.setBackground(Color.BLUE);
		window.add(backgroundPanel);
	\end{lstlisting}

  More advanced panels are available (JSplitPane, JScrollPane, JTabbedPane etc..)
\end{frame}

\subsection{Menus}
\begin{frame}[fragile]{Menus}
  \texttt{Menus} are used to display multiple possible actions to the user.
	\begin{lstlisting}
		// Menu Bar holds all menus
    MenuBar bar = new MenuBar();
    
    // Create new menu which holds menu items and submenus
		Menu menu = new Menu("File");

    // Create new menu item 
		MenuItem item = new MenuItem("Create new User");
    
    // Combine everything
		menu.add(item);
		bar.add(menu);
		window.setMenuBar(bar);
	\end{lstlisting}
\end{frame}

\subsection{Actions}
\begin{frame}[fragile]{Actions}
  In order to respond to button presses or other changes in the UI, you need to add listeners
	\begin{lstlisting}
    // Create menu item
		MenuItem item = new MenuItem("Create new User");
  
    // Add listener
		item.addActionListener(new ActionListener() {
			@Override
			public void actionPerformed(ActionEvent e) {
				userManager.addUser();
			}
		});
  \end{lstlisting}
\end{frame}
    
\begin{frame}[fragile]{Actions}
  Another example for a list
	\begin{lstlisting}
    // Create list with single selection option.
		this.userList = new JList();
		this.userList.setSelectionMode(ListSelectionModel.SINGLE_SELECTION);
      
    // Add listener
		this.userList.addListSelectionListener(new ListSelectionListener() {
			@Override
			public void valueChanged(ListSelectionEvent e) {
				int selectedIndex = this.userList.getSelectedIndex();
			}
		});
    \end{lstlisting}
\end{frame}
\end{document}