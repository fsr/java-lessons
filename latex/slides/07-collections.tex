%document
\documentclass[10pt]{beamer}
%theme
\usetheme{metropolis}
% packages
\usepackage{color}
\usepackage{listings}
\usepackage[ngerman]{babel}
\usepackage[utf8]{inputenc}
\usepackage{multicol}


% color definitions
\definecolor{mygreen}{rgb}{0,0.6,0}
\definecolor{mygray}{rgb}{0.5,0.5,0.5}
\definecolor{mymauve}{rgb}{0.58,0,0.82}

\lstset{language=Java,
	basicstyle=\ttfamily\footnotesize,
	keywordstyle=\color{purple},
	commentstyle=\color{darkgreen},
	numberstyle=\tiny\color{gray},
	stringstyle=\color{blue},
	tabsize=4,
	showstringspaces=false,
	breaklines=true,
	keepspaces=true,
	numbers=left,
	escapechar=@
}

\def\ContinueLineNumber{\lstset{firstnumber=last}}
\def\StartLineAt#1{\lstset{firstnumber=#1}}
\let\numberLineAt\StartLineAt



\newcommand{\codeline}[1]{
	\alert{\texttt{#1}}
}

% This Document contains the information about this course.

% Authors of the slides
\author{Felix Döring, Felix Wittwer}

% Name of the Course
\institute{Java-Kurs}

% Fancy Logo
\titlegraphic{\hfill\includegraphics[height=1.25cm]{../templates/fsr_logo_cropped}}


\title{Java}
\subtitle{Collections}
\date{\today}

\begin{document}

\begin{frame}
\titlepage
\end{frame}

\begin{frame}{Overview}
\tableofcontents
\end{frame}

\section{Generics}
\subsection{What is a generic}
\begin{frame}[fragile]{Generics}
	\begin{lstlisting}
		Object myStringAsObject = "klaus";
		String myStringAsString = (String) myStringAsObject;
	\end{lstlisting}
\end{frame}

\begin{frame}[fragile]{Generics}
	\begin{lstlisting}
		Object myStringAsObject = Integer.valueOf("42");
		String myStringAsString = (String) myStringAsObject;
	\end{lstlisting}
\end{frame}

\begin{frame}[fragile]{Why it won't work:}
		Integer can't be casted to String.
		
		The Code before will compile but still cause an Exception in the JVM.
\end{frame}

\begin{frame}[fragile]{Generics}
	\begin{lstlisting}[basicstyle=\ttfamily\scriptsize]
		public class Box {
			private Object object;

			public void set(Object object) { this.object = object; }
			public Object get() { return object; }
		}

	\end{lstlisting}
\end{frame}

\begin{frame}[fragile]{Generics}
	\begin{lstlisting}[basicstyle=\ttfamily\scriptsize]
		public class Box<T> {
		    // T stands for "Type"
		    private T t;

		    public void set(T t) { this.t = t; }
		    public T get() { return t; }
		}
		
		Box<Integer> integerBox; = new Box<Integer>();

	\end{lstlisting}
\end{frame}

\subsection{Wrapper Classes}

\begin{frame}{Wrapper Class}
	Primitive data types can not be elements in collections. 
	Use wrapper classes like \emph{Integer} instead.
	\begin{center}
		\begin{tabular}{ c  c }
			boolean & Boolean \\
			byte & Byte \\
			char & Character \\
			int & Integer \\
			float & Float \\
			double & Double \\
			long & Long \\
			short & Short
		\end{tabular}
	\end{center}
\end{frame}

\section{Collections}
\subsection{Overview}
\begin{frame}{Collections Framework}
	Java offers various data structures like \textbf{Sets}, \textbf{Lists} and \textbf{Maps}.
	Those structures are part of the collections framework.

	There are interfaces to access the data structures in an easy way.
	There are multiple implementations for various needs.
	Alternatively you can use your own implementations.
\end{frame}

\subsection{Set and List}
\begin{frame}[fragile]{Set}
	A set is a collection that holds one type of objects.
	A set can not contain one element twice.
	Like all collections the interface \emph{Set} is part of the package \texttt{java.util}.
	\begin{lstlisting}[basicstyle=\ttfamily\scriptsize]
	import java.util.*;

	public class TestSet {
	    
	    public static void main(String[] args) {
	        Set<String> set = new HashSet<String>();
	    
	        set.add("foo");
	        set.add("bar");
	        set.remove("foo");
	        System.out.println(set); // prints: [bar]
	    }
	}
	\end{lstlisting}
	In the following examples \texttt{import java.util.*;} will be omitted.
\end{frame}

\begin{frame}[fragile]{List}
	A list is an ordered collection.
	\vfill
	The implementation \texttt{LinkedList} is a double-linked list.
	\begin{lstlisting}[basicstyle=\ttfamily\scriptsize]
	public static void main(String[] args) {
	
	    List<String> list = new LinkedList<String>();
	    
	    list.add("foo"); 
	    list.add("foo"); // insert "foo" at the end
	    list.add("bar");
	    list.add("foo");
	    list.remove("foo"); // removes the first "foo"
	    
	    System.out.println(list); // prints: [foo, bar, foo]
	}
	\end{lstlisting}
\end{frame}
	
\begin{frame}[fragile]{List Methods}
	some useful List methods:\\
	\vspace{1em}
	\begin{tabular}{ r l r }
		void & \texttt{add(int index, E element)}
		& \footnotesize{insert element at position index} \\
		E &\texttt{get(int index)}
		& \footnotesize{get element at position index} \\
		E &\texttt{set(int index, E element)}
		& \footnotesize{replace element at position index} \\
		E &\texttt{remove(int index)}
		& \footnotesize{remove element at position index}
	\end{tabular}
	\vfill
	some useful LinkedList methods:\\
	\vspace{1em}
	\begin{tabular}{ r l r }
		void & \texttt{addFirst(E element)}
		& \footnotesize{append element to the beginning} \\
		E & \texttt{getFirst()}
		& \footnotesize{get first element} \\
		void & \texttt{addLast(E element)}
		& \footnotesize{append element to the end} \\
		E & \texttt{getLast()}
		& \footnotesize{get last element}
	\end{tabular}
\end{frame}

\subsection{Iterating}
\begin{frame}[fragile]{For Loop}
	The for loop can iterate over every element of a collection:\\
	\hspace{1em}\texttt{for (E e : collection)}
	\begin{lstlisting}
	public static void main(String[] args) {
	
	    List<Integer> list = 
	        new LinkedList<Integer>();
	    
	    list.add(1);
	    list.add(3);
	    list.add(3);
	    list.add(7);
	    
	    for (Integer i : list) {
	        System.out.print(i + " "); // prints: 1 3 3 7
	    }
	}
	\end{lstlisting}
\end{frame}

\begin{frame}[fragile]{Iterator}
	An iterator iterates step by step over a collection.
	\begin{lstlisting}[basicstyle=\ttfamily\scriptsize]
	public static void main(String[] args) {
	
	    List<Integer> list = new LinkedList<Integer>();
	    
	    list.add(1);
	    list.add(3);
	    list.add(3);
	    list.add(7);
	    
	    Iterator<Integer> iter = list.iterator();
	    
	    while (iter.hasNext()) {
	        System.out.print(iter.next());
	    }
	    // prints: 1337
	}
	\end{lstlisting}
\end{frame}

\begin{frame}[fragile]{Iterator}
	A standard iterator has only three methods:
	\begin{itemize}
	\item \texttt{boolean hasNext()} - indicates if therer are more elements
	\item \texttt{E next()} - returns the next element
	\item \texttt{void remove()} - returns the current element
	\end{itemize}
	\vspace{1em}
	The iterator is instanced via \texttt{collection.iterator()} :
	\begin{lstlisting}[basicstyle=\ttfamily\scriptsize]
	    Collection<E> collection = new Implementation<E>;
	    Iterator<E> iter = collection.iterator();
	\end{lstlisting}
	Special iterators like \emph{ListIterator} are more sophisticated.
\end{frame}

\subsection{Map}
\begin{frame}[fragile]{Map}
	The interface \emph{Map} is not a subinterface of \emph{Collection}.\\
	A map contains pairs of key and value. Each key refers to a value. 
	Two keys can refer to the same value. There are not two equal keys in one map.
	\emph{Map} is part of the package \texttt{java.util}.
	\vfill
	\begin{lstlisting}[basicstyle=\ttfamily\scriptsize]
	public static void main (String[] args) {
	
	    Map<Integer, String> map = 
	        new HashMap<Integer, String>();
	    
	    map.put(23, "foo");
	    map.put(28, "foo");
	    map.put(31, "bar");
	    map.put(23, "bar"); // "bar" replaces "foo" for key = 23
	    
	    System.out.println(map);
	    // prints: {23=bar, 28=foo, 31=bar}
	}
	\end{lstlisting}
\end{frame}

\begin{frame}[fragile]{Key, Set and Values}
	You can get the set of keys from the map.
	Because one value can exist multiple times a collection is used for the values.
	\begin{lstlisting}[basicstyle=\ttfamily\scriptsize]
	public static void main (String[] args) {
	
	    // [...] map like previous slide
	    
	    Set<Integer> keys = map.keySet();
	    Collection<String> values = map.values();
	    
	    System.out.println(keys);
	    // prints: [23, 28, 31]
	    
	    System.out.println(values);
	    // prints: [bar, foo, bar]
	}
	\end{lstlisting}
\end{frame}

\begin{frame}[fragile]{Iterator}
	To iterate over a map use the iterator from the set of keys.
	\begin{lstlisting}[basicstyle=\ttfamily\scriptsize]
	public static void main (String[] args) {
	
	    // [...] map, keys, values like previous slide    	    
	    Iterator<Integer> iter = keys.iterator();
	    
	    while(iter.hasNext()) {
	        System.out.print(map.get(iter.next()) + " ");
	    } // prints: bar foo bar
	    
	    System.out.println(); // print a line break
	    
	    for(Integer i: keys) {
	        System.out.print(map.get(i) + " ");
	    } // prints: bar foo bar
	}
	\end{lstlisting}
\end{frame}

\begin{frame}[fragile]{Nested Maps}
	Nested maps offer storage with key pairs.
	\begin{lstlisting}[basicstyle=\ttfamily\scriptsize]
	public static void main (String[] args) {		
	
	    Map<String, Map<Integer, String>> addresses = 
		    new HashMap<String, Map<Integer, String>>();
		
	    addresses.put("Noethnitzer Str.", 
	        new HashMap<Integer, String>());
		
	    addresses.get("Noethnitzer Str.").
	        put(46, "Andreas-Pfitzmann-Bau");
	    addresses.get("Noethnitzer Str.").
	        put(44, "Fraunhofer IWU");
	}
	\end{lstlisting}
\end{frame}

\begin{frame}[fragile]{Maps and Lambda}
	\begin{lstlisting}[basicstyle=\ttfamily\scriptsize]
		map.forEach((k,v) -> {
			//Key and Value
			System.out.println("Key: " + k + ", value: " v);
		})
	\end{lstlisting}
\end{frame}

\begin{frame}[fragile]{Maps and For Each}
  You can interate through the entry set of a map (available before Java 1.8)
	\begin{lstlisting}[basicstyle=\ttfamily\scriptsize]
      Map<String, String> map = ...
      for (Map.Entry<String, String> entry : map.entrySet()) {
        System.out.println("Key: " + entry.getKey() +
        ", value" + entry.getValue());
      }
	\end{lstlisting}
\end{frame}

\begin{frame}{Overview}
	\begin{center}
		\begin{tabular}{ l | l }
			List & Keeps order of objects \\
				 & Easily traversible \\
				 & Search not effective \\
			\hline
			Set  & No duplicates \\
				 & No order - still traversible \\
				 & Effective searching \\
			\hline
			Map  & Key-Value storage \\
				 & Search super-effective \\
				 & Traversing difficult
			
		\end{tabular}
	\end{center}
\end{frame}

\end{document}